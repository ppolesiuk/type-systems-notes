\usetikzlibrary{shapes}

\chapter{Semantics}

%%%%%%%%%%%%%%%%%%%%%%%%%%%%%%%%%%%%%%%%%%%%%%%%%%%%%%%%%%%%%%%%%%%%%%%%%%%%%%%
\section{Denotational Semantics}

Denotational semantics talks about the result of a computation, without specifying the process. We interpret terms as objects from some domain $\mathcal{D}$, which consists of mathematical structures designed to reflect meanings of programs. 

In the Logic course, we have already seen an example of denotational semantics, namely the denotational semantics of propositional calculus. Our terms were propositional formulae and our domain was the set $\{ \mathbf{T}, \mathbf{F} \}$. We discussed valuations $\sigma$ of propositional variables. For any such $\sigma$, we defined the valuation $\hat{\sigma}$ of formulae. For instance, if we needed to express that the formula $p \lor q$ is true under valuation $\sigma$, we wrote $\hat{\sigma}(p \lor q) = \mathbf{T}$. Switching to a different notation, one may write $\Denot{p \lor q}_\sigma = \mathbf{T}$.

Below, the environment $\rho$ is the equivalent of propositional $\sigma$. The function $\varphi$ turns denotations of lambda-abstractions into functions on denotations (limited in some reasonable way). The function $\psi$ works in the other way.

\begin{align*}
\Denot{\cdot} &\colon
  \syncatset{Term} \times (\syncatset{Var} \rightarrow \mathcal{D})
    \rightarrow \mathcal{D}_{\bot} \\
\varphi &\colon \mathcal{D} \rightarrow (\mathcal{D} \stackrel{\text{cont}}{\rightarrow} \mathcal{D}_\bot) \\
\psi &\colon (\mathcal{D} \stackrel{\text{cont}}{\rightarrow} \mathcal{D}_\bot) \rightarrow \mathcal{D}
\end{align*}

\begin{alignat*}{2}
  \Denot{x}_\rho & = \rho(x) \\
  \Denot{\lambda x.t}_\rho & =
    \psi(\mathbf{\lambda} v. \Denot{t}_{\rho [x \mapsto v]}) \\
  \Denot{t_1 \; t_2}_\rho & =
    \varphi(\Denot{t_1}_\rho) \Denot{t_2}_\rho
\end{alignat*}

%%%%%%%%%%%%%%%%%%%%%%%%%%%%%%%%%%%%%%%%%%%%%%%%%%%%%%%%%%%%%%%%%%%%%%%%%%%%%%%
\section{Expressions, Values, and Operational Semantics}
Operational semantics concerns the execution of programs. Here, it will come in handy to distinguish between expressions and values among terms, since values cannot undergo any further computation. 

\begin{alignat*}{2}
  \syncatset{Expr}  & \ni e && \Coloneqq v \mid e\;e         \tag{expressions} \\
  \syncatset{Value} & \ni v && \Coloneqq \lambda x.e \mid x  \tag{values}
\end{alignat*}

Operational semantics can be divided into big-step semantics and small-step semantics. The former defines a relation which signifies the evaluation of an expression to a value. It is, however, unable to say things about programs that never terminate. In contrast, the latter describes the elementary steps in the execution of a program, where an expression reduces to an expression which may or may not be a value.
%%%%%%%%%%%%%%%%%%%%%%%%%%%%%%%%%%%%%%%%%%%%%%%%%%%%%%%%%%%%%%%%%%%%%%%%%%%%%%%
\section{Structural Operational Semantics}
Structural operational semantics falls into the category of small-step semantics. We can classify evaluation rules to:
\begin{itemize}
    \item reduction rules, which do computations (e.g. application rule below),
    \item structural congruence rules, which indicate when and which subterms can be evaluated;
        they define evaluation order.
\end{itemize}

\begin{mathpar}
  \inferrule{\phantom{e}}
            {(\lambda x.e)\;v \longrightarrow e\{v/x\}}
  
  \inferrule{e_1 \longrightarrow e_1'}
            {e_1 \; e_2 \longrightarrow e_1' \; e_2}

  \inferrule{e \longrightarrow e'}
            {v \; e \longrightarrow v \; e'}
\end{mathpar}

The process of evaluation is defined unambiguously. The congruence rules allow us to deduce a computation step in a big, nested expression from a step in its subexpression. Unfortunately, the subexpression's surroundings, i.e. the context, are forgotten. This will become a disadvantage once we get to control flow operators such as \texttt{call/cc}.
%%%%%%%%%%%%%%%%%%%%%%%%%%%%%%%%%%%%%%%%%%%%%%%%%%%%%%%%%%%%%%%%%%%%%%%%%%%%%%%
\section{Evaluation Contexts and Reduction Semantics}
Reduction semantics is another kind of small-step semantics.
The evaluation context $E$ represents the rest of the computation during evaluation.
Current position is called the hole $\square$. Evaluation contexts' grammar
is a more compact representation of structural congruence rules.

\begin{mathpar}
  \inferrule{\phantom{e}}
            {(\lambda x.e) \; v \rightharpoonup e\{v/x\}}

  \inferrule{e_1 \rightharpoonup e_2}
            {E[e_1] \longrightarrow E[e_2]}
\end{mathpar}

\begin{alignat*}{2}
  \syncatset{Ectx} & \ni E && \Coloneqq \square \mid E \; e \mid v \; E
\end{alignat*}

Evaluation contexts can be represented in two styles:
\emph{outside-in} and \emph{inside-out}.
For this simple case (CBV lambda calculus) the grammar is the same,
but they differ in the \emph{plug} operation.

outside-in:
\begin{alignat*}{2}
  \square[e] & = e        \\
  (E\;e')[e] & = E[e]\;e' \\
  (v\;E)[e]  & = v\;E[e]
\end{alignat*}

inside-out:
\begin{alignat*}{2}
  \square[e] & = e       \\
  (E\;e')[e] & = E[e e'] \\
  (v\;E)[e]  & = E[v e]
\end{alignat*}

When we associate evaluation contexts with stack frames, then these two variants
correspond to two different orderings. In the inside-out style, the root contains
the first operation that will be applied to the value produced by $\square$. \\

Figures below show these contexts for $1 + (\square - 2)$.

\begin{figure}[h]
    \tikzset{
        node/.style={draw,minimum size=0.8cm, circle},
        hole/.style={draw,minimum size=0.8cm, regular polygon, regular polygon sides=4}
    }

    \begin{minipage}{0.45\textwidth}
        \centering
        \begin{tikzpicture}[scale=0.02]
            \node [node] (1) at (51, 179) {$+$};
            \node [node] (2) at (21, 100) {1};
            \node [node] (3) at (82, 100) {$-$};
            \node [hole] (4) at (51,  21) {};
            \node [node] (5) at (112, 21) {2};

            \draw (1) -- (2);
            \draw (1) -- (3);
            \draw (3) -- (4);
            \draw (3) -- (5);
        \end{tikzpicture}
        \caption{Outside-in context}
    \end{minipage}
    \hfill
    \begin{minipage}{0.45\textwidth}
        \centering
        \begin{tikzpicture}[scale=0.02]
            \node [node] (1) at (51, 179) {$-$};
            \node [node] (2) at (21, 100) {$+$};
            \node [node] (3) at (82, 100) {2};
            \node [node] (4) at (-10, 21) {1};
            \node [hole] (5) at (51, 21)  {};

            \draw (1) -- (2);
            \draw (1) -- (3);
            \draw (2) -- (4);
            \draw (2) -- (5);
        \end{tikzpicture}
        \caption{Inside-out context}
    \end{minipage}
\end{figure}

%%%%%%%%%%%%%%%%%%%%%%%%%%%%%%%%%%%%%%%%%%%%%%%%%%%%%%%%%%%%%%%%%%%%%%%%%%%%%%%
\section{Programs, and Non-local Reductions: \texttt{call/cc}}
\label{sec:non-local-red}

\begin{alignat*}{2}
  \syncatset{Prog} & \ni p && \Coloneqq e
    \tag{programs} \\
  \syncatset{Expr} & \ni e && \Coloneqq v
    \mid e\;e \mid \mathcal{K} x.e \mid \mathcal{A} p
    \tag{expressions} \\
  \syncatset{Value} & \ni v && \Coloneqq x \mid v
    \tag{values}
\end{alignat*}

Thanks to evaluation contexts, we can define non-local control flow constructs.

\begin{alignat*}{2}
  E[\mathcal{K} x.e] & \longrightarrow E[e\{\lambda y.\mathcal{A} E[y] / x\}] \\
  E[\mathcal{A} p]   & \longrightarrow p
\end{alignat*}

%%%%%%%%%%%%%%%%%%%%%%%%%%%%%%%%%%%%%%%%%%%%%%%%%%%%%%%%%%%%%%%%%%%%%%%%%%%%%%%
\section{Further Reading}

As pointed out during the lecture, we've already seen one-hole contexts
on the Functional Programming course
in the data structure called \emph{The Zipper}~\citep{Huet97}.
Interestingly, \citet{McBride} observed that one-hole contexts used
in the Zipper data structure can be derived automatically,
and such a derivation has a lot in common with
a symbolic derivative of a function.
