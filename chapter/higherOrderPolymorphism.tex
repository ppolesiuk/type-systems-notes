\chapter{Higher Order Polymorphism}

%%%%%%%%%%%%%%%%%%%%%%%%%%%%%%%%%%%%%%%%%%%%%%%%%%%%%%%%%%%%%%%%%%%%%%%%%%%%%%%
\section{System F$_\omega$}

We would like to add a new feature to our calculi: type functions, known as type operators.
These operators will enable us to use convenient type aliases and recursive type constructors
like $\text{List}$, which, for each type $T$, returns a type of list containing elements of type $T$.
At the level of types, we add lambda abstraction and application. This addition creates
possibility to write meaningless types like $(Unit\; Unit)$, so we create a type system
for types, called kinds, which is basically a copy of simply typed lambda calculus.
As we can write the same type in different ways, like $Nat$ and $(Id\; Nat)$, we establish
an equivalence relation on types.

\begin{alignat*}{2}
  \syncatset{Kind} & \ni \kappa && \Coloneqq * \mid \kappa \to \kappa \\
  \syncatset{Type} & \ni \tau   && \Coloneqq \alpha \mid \tau \to \tau \mid
    \forall\alpha^\kappa.\tau \mid \tau\;\tau \mid \lambda\alpha^\kappa.\tau \mid \dots \\
  \syncatset{Expr} & \ni e      && \Coloneqq v \mid e\;e \mid e\;* \\
  \syncatset{Value} & \ni v     && \Coloneqq x \mid \lambda x.e \mid \Lambda e
\end{alignat*}

Alternative type formulation (with explicit kinds):
\begin{alignat*}{2}
  \tau^k & \Coloneqq \alpha^\kappa \mid \tau^{\kappa'\to\kappa}\;\tau^{\kappa'} \\
  \tau^* & \Coloneqq \dots \mid \tau^* \to \tau^* \mid \forall\alpha^\kappa.\tau^* \\
  \tau^{\kappa_1\to\kappa_2} & \Coloneqq \dots \mid \lambda\alpha^{\kappa_1}.\tau^{\kappa_2}
\end{alignat*}

Kind system:
\begin{mathpar}
  \inferrule{(\alpha :: \kappa) \in \Delta}
            {\Delta \vdash \alpha :: \kappa}

  \inferrule{\Delta, \alpha :: \kappa_1 \vdash \tau :: \kappa_2}
            {\Delta \vdash \lambda\alpha^{\kappa_1}.\tau :: \kappa_1 \to \kappa_2}

  \inferrule{\Delta \vdash \tau_2 :: \kappa_1 \to \kappa_2 \\ \Delta \vdash \tau_1 :: \kappa_1}
            {\Delta \vdash \tau_2\;\tau_1 :: \kappa_2}

  \inferrule{\Delta \vdash \tau_1 :: * \\ \Delta \vdash \tau_2 :: *}
            {\Delta \vdash \tau_1 \to \tau_2 :: *}

  \inferrule{\Delta, \alpha :: \kappa \vdash \tau :: *}
            {\Delta \vdash \forall\alpha^\kappa.\tau :: *}
\end{mathpar}

Type system:
\begin{mathpar}
  \inferrule{\Gamma, x:\tau_1;\Delta \vdash e : \tau_2 \\ \Gamma \vdash \tau_1 :: *}
            {\Gamma;\Delta \vdash \lambda x.e : \tau_1 \to \tau_2}

  \inferrule{\Gamma;\Delta \vdash e : \forall\alpha^\kappa.\tau \\ \Delta \vdash \tau' :: \kappa}
            {\Gamma;\Delta \vdash e\;* : \tau\{\tau'/\alpha\}}

  \inferrule{\Gamma;\Delta \vdash e : \tau \\ \Gamma \vdash \tau \equiv \tau'}
            {\Gamma;\Delta \vdash e : \tau'}

  \inferrule{\Gamma;\Delta, \alpha::\kappa \vdash e : \tau}
            {\Gamma;\Delta \vdash \Lambda e : \forall \alpha^\kappa\ldotp \tau}

  \inferrule{(x : \tau) \in \Gamma}
            {\Gamma;\Delta \vdash x : \tau}
\end{mathpar}

$\Delta \vdash \cdot \equiv \cdot$ is the smallest congruence containing the rule
\begin{mathpar}
  \inferrule{\Delta \vdash \tau' :: \kappa}
            {\Delta \vdash (\lambda\alpha^\kappa.\tau)\tau' \equiv \tau\{\tau'/\alpha\}}
\end{mathpar}

%%%%%%%%%%%%%%%%%%%%%%%%%%%%%%%%%%%%%%%%%%%%%%%%%%%%%%%%%%%%%%%%%%%%%%%%%%%%%%%
\section{Problems with Type Reconstruction}

%%%%%%%%%%%%%%%%%%%%%%%%%%%%%%%%%%%%%%%%%%%%%%%%%%%%%%%%%%%%%%%%%%%%%%%%%%%%%%%
\section{Relational Model}

\begin{alignat*}{2}
  \Denot{*} & = \mathcal{P}(\syncatset{Value}_{\varnothing}) \\
  \Denot{\kappa_1\to\kappa_2} & = \Denot{\kappa_2}^{\Denot{\kappa_1}} \\
  \Denot{\Delta} & = \Pi(\alpha::\kappa)\in\Delta. \Denot{\kappa} \\
  \Denot{\alpha^{\kappa}}_\eta & = \eta(\alpha) \\
  \Denot{\tau_1 \to \tau_2}_\eta &=
    \{ v \mid \forall v' \in \Denot{\tau_1}_\eta\ldotp v\;v' \in \mathcal{E}\Denot{\tau_2}_\eta \} \\
  \Denot{\forall \alpha^\kappa \ldotp \tau}_{\eta} &= 
    \{ v \mid \forall R \in \Denot{\kappa}_\eta \ldotp
    v\;* \in \mathcal{E}\Denot{\tau}_{\eta[\alpha \mapsto R]} \} \\
  \Denot{\tau_1^{\kappa_1 \to \kappa_2} \tau_2^{\kappa_1}}_\eta & =
    \Denot{\tau_1}_\eta \Denot{\tau_2}_\eta \\
  \Denot{\lambda \alpha^ \kappa \ldotp \tau }_\eta &=
    \lambda R \in \Denot{\kappa}_\eta \ldotp \Denot{\tau}_{\eta[\alpha \mapsto R]}
\end{alignat*}

%%%%%%%%%%%%%%%%%%%%%%%%%%%%%%%%%%%%%%%%%%%%%%%%%%%%%%%%%%%%%%%%%%%%%%%%%%%%%%%
\section{Iso-recursive Types of Higher Kinds}

\[
  \tau \Coloneqq \dots \mid \mu\alpha^\kappa.\tau
\]

\begin{mathpar}
  \inferrule{\Gamma, \alpha :: \kappa \vdash \tau :: \kappa}
            {\Gamma \vdash \mu\alpha^\kappa.\tau :: \kappa}
\end{mathpar}

List $= \lambda\alpha^*.\mu\beta.1+\alpha\times\beta$\\
NList $= \mu\beta^{*\to*}.\lambda\alpha^*. 1 + \beta(\alpha \times \alpha) + \alpha \times \beta(\alpha \times \alpha)$

%%%%%%%%%%%%%%%%%%%%%%%%%%%%%%%%%%%%%%%%%%%%%%%%%%%%%%%%%%%%%%%%%%%%%%%%%%%%%%%
\section{Further Reading}
