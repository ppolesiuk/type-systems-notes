\chapter{Subtyping}

%%%%%%%%%%%%%%%%%%%%%%%%%%%%%%%%%%%%%%%%%%%%%%%%%%%%%%%%%%%%%%%%%%%%%%%%%%%%%%%
\section{Structural Subtyping}

In this chapter we will model something that
we already know from many programming languages.
For example in C we can use variable of type \texttt{int}
in place when \texttt{float} is needed.
We will say that \texttt{int} is a \emph{subtype} of float
and write it as $\texttt{int} <: \texttt{float}$.
In Java we have class inheritance:
\begin{verbatim}
  class B extends A {

  }
\end{verbatim}
An object of class \texttt{B} can be used as an object of class \texttt{A}.
We can also look at modules in OCaml.
A module of signature
\begin{verbatim}
  sig
    val foo : t1
    val bar : t2
  end
\end{verbatim}
is smaller than a module of signature
\begin{verbatim}
  sig
    val foo: t1
  end
\end{verbatim}
The former can be used in any place that the latter is needed.
As always we will define our new relation $<:$ using inference rules.
This relation should be reflexive and transitive.
\begin{mathpar}
  \inferrule{ }
            {\tau <: \tau}

  \inferrule{\tau_1 <: \tau_2 \\ \tau_2 <: \tau_3}
            {\tau_1 <: \tau_3}
\end{mathpar}

We can also add $\bot$ and $\top$ as the smallest and biggest types.
\begin{mathpar}
  \inferrule{ }
            {\bot <: \tau}

  \inferrule{ }
            {\tau <: \top}
\end{mathpar}

Along with structural rules for arrow and product.
\begin{mathpar}
  \inferrule{\tau_1' <: \tau_1 \\ \tau_2 <: \tau_2'}
            {\tau_1 \to \tau_2 <: \tau_1' \to \tau_2'}

  \inferrule{\tau_1 <: \tau_1' \\ \tau_2 <: \tau_2'}
            {\tau_1 \times \tau_2 <: \tau_1' \times \tau_2'}
\end{mathpar}
One thing to note here is that for arrow
types at the negative positions need to be checked in the reverse order
than those at the positive position.

Similarly when we would want to establish subtyping for mutable cells
we need to both be able to read and write to them,
which gives as a following inference rule
\begin{mathpar}
  \inferrule{\tau <: \tau' \\ \tau' <: \tau}
            {\tau \texttt{ ref} <: \tau' \texttt{ ref}}.
\end{mathpar}

Lastly following our intuition about subtyping,
that anything of a smaller type can be used when a bigger type is needed,
we add new typing rule called \emph{subsumption}.
\begin{mathpar}
  \inferrule{\Gamma \vdash e : \tau_1 \\ \tau_1 <: \tau_2}
            {\Gamma \vdash e : \tau_2}
\end{mathpar}

%%%%%%%%%%%%%%%%%%%%%%%%%%%%%%%%%%%%%%%%%%%%%%%%%%%%%%%%%%%%%%%%%%%%%%%%%%%%%%%
\section{Algorithmic Subtyping}

When we extend our calculi one question that one should ask is
what about type reconstruction.
When looking at our new subsumption rule,
we can see that there may be some problems.
This rule is not controlled by syntax and
it is not clear when it should be used.

To resolve this problem we can try inferring
the smallest possible type for a expression.
Similarly to \autoref{ch:type-reconstruction}
we will define two operations, namely bottom-up and top-down.
\begin{mathpar}
  \inferrule{(x : \tau) \in \Gamma}
            {\Gamma \vdash x \Uparrow \tau}

  \inferrule{\Gamma, x : \tau_1 \vdash e \Uparrow \tau_2}
            {\Gamma \vdash \lambda x : \tau_1 . e \Uparrow \tau_1 \to \tau_2}

  \inferrule{\Gamma \vdash e_1 \Uparrow \tau_2 \to \tau_1 \\ \Gamma \vdash e_2 \Downarrow \tau_2}
            {\Gamma \vdash e_1\;e_2 \Uparrow \tau_1}

  \inferrule{\Gamma \vdash e_1 \Uparrow \bot \\ \Gamma \vdash e_2 \Uparrow \tau}
            {\Gamma \vdash e_1\;e_2 \Uparrow \bot}

  \inferrule{\Gamma, x : \tau_1 \vdash e \Downarrow \tau_2}
            {\Gamma \vdash \lambda x.e \Downarrow \tau_1 \to \tau_2}

  \inferrule{\Gamma \vdash e \Uparrow \tau' \\ \tau' <: \tau}
            {\Gamma \vdash e \Downarrow \tau}

  \inferrule{\Gamma \vdash e_1 \Downarrow \texttt{Bool} \\ \Gamma \vdash e_2 \Uparrow \tau_2 \\ \Gamma \vdash e_3 \Uparrow \tau_3}
            {\Gamma \vdash \texttt{if}\;e_1\;\texttt{then}\;e_2\;\texttt{else}\;e_3 \Uparrow \tau_2 \sqcup \tau_3}
\end{mathpar}
Most of those rules are pretty self-explanatory.
In order to properly write inference rule for \texttt{if}
we need to define two operations---\emph{join} and \emph{meet}
denoted accordingly as $\sqcup$ and $\sqcap$.
They will give as a supremum and infimum of their arguments.
Here we give part of their definitions. The rest should be rather obvious.
\begin{alignat*}{2}
  \bot \sqcup \tau & = \tau \\
  \top \sqcup \tau & = \top \\
  (\tau_1 \times \tau_2) \sqcup (\tau_1' \times \tau_2') & =
    (\tau_1 \sqcup \tau_1') \times (\tau_2 \sqcup \tau_2') \\
  (\tau_1 \to \tau_2) \sqcup (\tau_1' \to \tau_2') & =
    (\tau_1 \sqcap \tau_1') \to (\tau_2 \sqcup \tau_2') \\
  (\tau_1 \times \tau_2) \sqcup (\tau_1' \to \tau_2') & = \top \\
  \cdots \\
  \bot \sqcap \tau & = \bot \\
  \top \sqcap \tau & = \tau \\
  \cdots
\end{alignat*}

The relation between $\Gamma \vdash e : \tau$
and $\Gamma \vdash e \Uparrow \tau$ can be summed up in the following theorem.

\begin{theorem}{\ }
  \begin{thmenumerate}
   \item If $\Gamma \vdash e \Uparrow \tau$, then $\Gamma \vdash e : \tau$.
   \item If $\Gamma \vdash e : \tau$, then $\Gamma \vdash e \Uparrow \tau$'
     for some $\tau' <: \tau$.
 \end{thmenumerate}
\end{theorem}

%%%%%%%%%%%%%%%%%%%%%%%%%%%%%%%%%%%%%%%%%%%%%%%%%%%%%%%%%%%%%%%%%%%%%%%%%%%%%%%
\section{Coercion Semantics}

Language with explicit coercions:

\begin{alignat*}{2}
  e & \Coloneqq \dots \mid c\;e \\
  v & \Coloneqq \dots \mid \langle c_1 \to c_2 \rangle\;v \\
  c & \Coloneqq c \to c \mid \bot \mid id \mid \top \mid c_{int}^{float} \mid c \circ c
\end{alignat*}

Every coercion $c$ corresponds to one of the subtyping rules:

\begin{mathpar}
  \inferrule{ }
            {id : \tau <: \tau}

  \inferrule{c_1 : \tau_1 <: \tau_2 \\ c_2 : \tau_2 <: \tau_3}
            {c_1 \circ c_2 : \tau_2 <: \tau_3}

  \inferrule{ }
            {\bot : \bot <: \tau}

  \inferrule{ }
            {\top : \tau <: \top}

  \inferrule{c_1 : \tau_1' <: \tau_1 \\ c_2 : \tau_2 <: \tau_2'}
            {c_1 \to c_2 : \tau_1 \to \tau_2 <: \tau_1' \to \tau_2'}

  \inferrule{ }
            {c_{int}^{float} : int <: float}
\end{mathpar}

Reduction semantics:

\begin{alignat*}{2}
  id\;v & \rightharpoonup v \\
  (c_1 \circ c_2)\;v & \rightharpoonup c_1 (c_2\;v) \\
  \top v & \rightharpoonup \langle\rangle \\
  (c_1 \to c_2)\;v & \rightharpoonup \langle c_1 \to c_2 \rangle\;v \\
  \langle c_1 \to c_2 \rangle\;v_1\;v_2 & \rightharpoonup c_2 (v_1 (c_1\;v_2))
\end{alignat*}

Translation from calculus with subtyping to calculus with explicit coercions:

\begin{mathpar}
  \inferrule{\Gamma, x : \tau_1 \vdash e : \tau_2 \rightsquigarrow e'}
            {\Gamma \vdash \lambda x.e : \tau_1 \to \tau_2 \rightsquigarrow \lambda x.e'}

  \inferrule{\Gamma \vdash e : \tau_1 \rightsquigarrow e' \\ \tau_1 <: \tau_2 \rightsquigarrow c}
            {\Gamma \vdash e : \tau_2 \rightsquigarrow c\;e'}
\end{mathpar}

%%%%%%%%%%%%%%%%%%%%%%%%%%%%%%%%%%%%%%%%%%%%%%%%%%%%%%%%%%%%%%%%%%%%%%%%%%%%%%%
\section{Coherence and Interlanguage Logical Relations}

%%%%%%%%%%%%%%%%%%%%%%%%%%%%%%%%%%%%%%%%%%%%%%%%%%%%%%%%%%%%%%%%%%%%%%%%%%%%%%%
\section{Polymorphism and Bounded Quantification}

\[
  \tau \Coloneqq \dots \mid \forall\alpha <: \tau.\tau
\]

Typing rules:

\begin{mathpar}
  \inferrule{\Delta, \alpha <: \tau_1; \Gamma \vdash e : \tau_2}
            {\Delta; \Gamma \vdash \Lambda e : \forall\alpha <: \tau_1.\tau_2}

  \inferrule{\Delta; \Gamma \vdash e : \forall\alpha <: \tau_1.\tau_2 \\ \Delta \vdash \tau <:\tau_1\{\tau/\alpha\}}
            {\Delta; \Gamma \vdash e\;* : \tau_2\{\tau/\alpha\}}
\end{mathpar}

Subtyping rules:

\begin{mathpar}
  \inferrule{(\alpha <: \tau) \in \Delta}
            {\Delta \vdash \alpha <: \tau}

  \inferrule{\Delta, \alpha <: \tau_1' \vdash \tau_1' <: \tau_1 \\ \Delta, \alpha <: \tau_1' \vdash \tau_2 <: \tau_2'}
            {\Delta \vdash \forall\alpha <: \tau_1.\tau_2 <: \forall\alpha <: \tau_1'.\tau_2'}
\end{mathpar}


%%%%%%%%%%%%%%%%%%%%%%%%%%%%%%%%%%%%%%%%%%%%%%%%%%%%%%%%%%%%%%%%%%%%%%%%%%%%%%%
\section{Further Reading}
