\chapter{Subtyping}

%%%%%%%%%%%%%%%%%%%%%%%%%%%%%%%%%%%%%%%%%%%%%%%%%%%%%%%%%%%%%%%%%%%%%%%%%%%%%%%
\section{Structural Subtyping}

Subtyping rules:

\begin{mathpar}
  \inferrule{ }
            {\tau <: \tau}

  \inferrule{\tau_1 <: \tau_2 \\ \tau_2 <: \tau_3}
            {\tau_1 <: \tau_3}

  \inferrule{ }
            {\bot <: \tau}

  \inferrule{ }
            {\tau <: \top}

  \inferrule{\tau_1' <: \tau_1 \\ \tau_2 <: \tau_2'}
            {\tau_1 \to \tau_2 <: \tau_1' \to \tau_2'}

  \inferrule{\tau_1 <: \tau_1' \\ \tau_2 <: \tau_2'}
            {\tau_1 \times \tau_2 <: \tau_1' \times \tau_2'}
\end{mathpar}

New typing rule (subsumption):

\begin{mathpar}
  \inferrule{\Gamma \vdash e : \tau_1 \\ \tau_1 <: \tau_2}
            {\Gamma \vdash e : \tau_2}
\end{mathpar}

%%%%%%%%%%%%%%%%%%%%%%%%%%%%%%%%%%%%%%%%%%%%%%%%%%%%%%%%%%%%%%%%%%%%%%%%%%%%%%%
\section{Algorithmic Subtyping}

\begin{mathpar}
  \inferrule{(x : \tau) \in \Gamma}
            {\Gamma \vdash x \Uparrow \tau}

  \inferrule{\Gamma, x : \tau_1 \vdash e \Uparrow \tau_2}
            {\Gamma \vdash \lambda x : \tau_1 . e \Uparrow \tau_1 \to \tau_2}

  \inferrule{\Gamma \vdash e_1 \Uparrow \tau_2 \to \tau_1 \\ \Gamma \vdash e_2 \Downarrow \tau_2}
            {\Gamma \vdash e_1\;e_2 \Uparrow \tau_1}

  \inferrule{\Gamma \vdash e_1 \Uparrow \bot \\ \Gamma \vdash e_2 \Uparrow \tau}
            {\Gamma \vdash e_1\;e_2 \Uparrow \bot}

  \inferrule{\Gamma, x : \tau_1 \vdash e \Downarrow \tau_2}
            {\Gamma \vdash \lambda x.e \Downarrow \tau_1 \to \tau_2}

  \inferrule{\Gamma \vdash e \Uparrow \tau' \\ \tau' <: \tau}
            {\Gamma \vdash e \Downarrow \tau}

  \inferrule{\Gamma \vdash e_1 \Downarrow \texttt{Bool} \\ \Gamma \vdash e_2 \Uparrow \tau_2 \\ \Gamma \vdash e_3 \Uparrow \tau_3}
            {\Gamma \vdash \texttt{if}\;e_1\;\texttt{then}\;e_2\;\texttt{else}\;e_3 \Uparrow \tau_2 \sqcup \tau_3}
\end{mathpar}

\begin{alignat*}{2}
  \bot \sqcup \tau & = \tau \\
  \top \sqcup \tau & = \top \\
  (\tau_1 \times \tau_2) \sqcup (\tau_1' \times \tau_2') & = (\tau_1 \sqcup \tau_1') \times (\tau_2 \sqcup \tau_2') \\
  (\tau_1 \to \tau_2) \sqcup (\tau_1' \to \tau_2') & = (\tau_1 \sqcap \tau_1') \to (\tau_2 \sqcup \tau_2') \\
  (\tau_1 \times \tau_2) \sqcup (\tau_1' \to \tau_2') & = \top
\end{alignat*}

%%%%%%%%%%%%%%%%%%%%%%%%%%%%%%%%%%%%%%%%%%%%%%%%%%%%%%%%%%%%%%%%%%%%%%%%%%%%%%%
\section{Coercion Semantics}

Language with explicit coercions:

\begin{alignat*}{2}
  e & \Coloneqq \dots \mid c\;e \\
  v & \Coloneqq \dots \mid \langle c_1 \to c_2 \rangle\;v \\
  c & \Coloneqq c \to c \mid \bot \mid id \mid \top \mid c_{int}^{float} \mid c \circ c
\end{alignat*}

Every coercion $c$ corresponds to one of the subtyping rules:

\begin{mathpar}
  \inferrule{ }
            {id : \tau <: \tau}

  \inferrule{c_1 : \tau_1 <: \tau_2 \\ c_2 : \tau_2 <: \tau_3}
            {c_1 \circ c_2 : \tau_2 <: \tau_3}

  \inferrule{ }
            {\bot : \bot <: \tau}

  \inferrule{ }
            {\top : \tau <: \top}

  \inferrule{c_1 : \tau_1' <: \tau_1 \\ c_2 : \tau_2 <: \tau_2'}
            {c_1 \to c_2 : \tau_1 \to \tau_2 <: \tau_1' \to \tau_2'}

  \inferrule{ }
            {c_{int}^{float} : int <: float}
\end{mathpar}

Reduction semantics:

\begin{alignat*}{2}
  id\;v & \rightharpoonup v \\
  (c_1 \circ c_2)\;v & \rightharpoonup c_1 (c_2\;v) \\
  \top v & \rightharpoonup \langle\rangle \\
  (c_1 \to c_2)\;v & \rightharpoonup \langle c_1 \to c_2 \rangle\;v \\
  \langle c_1 \to c_2 \rangle\;v_1\;v_2 & \rightharpoonup c_2 (v_1 (c_1\;v_2))
\end{alignat*}

Translation from calculus with subtyping to calculus with explicit coercions:

\begin{mathpar}
  \inferrule{\Gamma, x : \tau_1 \vdash e : \tau_2 \rightsquigarrow e'}
            {\Gamma \vdash \lambda x.e : \tau_1 \to \tau_2 \rightsquigarrow \lambda x.e'}

  \inferrule{\Gamma \vdash e : \tau_1 \rightsquigarrow e' \\ \tau_1 <: \tau_2 \rightsquigarrow c}
            {\Gamma \vdash e : \tau_2 \rightsquigarrow c\;e'}
\end{mathpar}

%%%%%%%%%%%%%%%%%%%%%%%%%%%%%%%%%%%%%%%%%%%%%%%%%%%%%%%%%%%%%%%%%%%%%%%%%%%%%%%
\section{Coherence and Interlanguage Logical Relations}

%%%%%%%%%%%%%%%%%%%%%%%%%%%%%%%%%%%%%%%%%%%%%%%%%%%%%%%%%%%%%%%%%%%%%%%%%%%%%%%
\section{Polymorphism and Bounded Quantification}

\[
  \tau \Coloneqq \dots \mid \forall\alpha <: \tau.\tau
\]

Typing rules:

\begin{mathpar}
  \inferrule{\Delta, \alpha <: \tau_1; \Gamma \vdash e : \tau_2}
            {\Delta; \Gamma \vdash \Lambda e : \forall\alpha <: \tau_1.\tau_2}

  \inferrule{\Delta; \Gamma \vdash e : \forall\alpha <: \tau_1.\tau_2 \\ \Delta \vdash \tau <:\tau_1\{\tau/\alpha\}}
            {\Delta; \Gamma \vdash e\;* : \tau_2\{\tau/\alpha\}}
\end{mathpar}

Subtyping rules:

\begin{mathpar}
  \inferrule{(\alpha <: \tau) \in \Delta}
            {\Delta \vdash \alpha <: \tau}

  \inferrule{\Delta, \alpha <: \tau_1' \vdash \tau_1' <: \tau_1 \\ \Delta, \alpha <: \tau_1' \vdash \tau_2 <: \tau_2'}
            {\Delta \vdash \forall\alpha <: \tau_1.\tau_2 <: \forall\alpha <: \tau_1'.\tau_2'}
\end{mathpar}


%%%%%%%%%%%%%%%%%%%%%%%%%%%%%%%%%%%%%%%%%%%%%%%%%%%%%%%%%%%%%%%%%%%%%%%%%%%%%%%
\section{Further Reading}
