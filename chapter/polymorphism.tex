\chapter{Parametric Polymorphism}

%%%%%%%%%%%%%%%%%%%%%%%%%%%%%%%%%%%%%%%%%%%%%%%%%%%%%%%%%%%%%%%%%%%%%%%%%%%%%%%
\section{System F}


\begin{alignat*}{2}
  \syncatset{Type} & \ni \tau && \Coloneqq
    \tau \to \tau \mid \forall\alpha.\tau \mid \alpha \tag{types}\\
  \syncatset{Expr} & \ni e && \Coloneqq
    v \mid e\;e \mid e\;* \tag{expressions} \\
  \syncatset{Value} & \ni v && \Coloneqq
    \lambda x.e \mid x \mid \Lambda e \tag{values}
\end{alignat*}

\begin{mathpar}
  \inferrule{\Delta,\alpha;\Gamma\vdash e : \tau}
            {\Delta;\Gamma\vdash \Lambda e : \forall\alpha.\tau}

  \inferrule{\Delta;\Gamma\vdash e : \forall\alpha.\tau}
            {\Delta;\Gamma\vdash e\;* : \tau\{\tau'/\alpha\}}
\end{mathpar}

Implicit alternative:

\begin{mathpar}
  \inferrule{\Delta,\alpha;\Gamma\vdash e : \tau}
            {\Delta;\Gamma\vdash e : \forall\alpha.\tau}

  \inferrule{\Delta;\Gamma\vdash e : \forall\alpha.\tau}
            {\Delta;\Gamma\vdash e : \tau\{\tau'/\alpha\}}
\end{mathpar}

Value restriction:

\begin{mathpar}
  \inferrule{\Delta,\alpha;\Gamma\vdash v : \tau}
            {\Delta;\Gamma\vdash v : \forall\alpha.\tau}
\end{mathpar}

%%%%%%%%%%%%%%%%%%%%%%%%%%%%%%%%%%%%%%%%%%%%%%%%%%%%%%%%%%%%%%%%%%%%%%%%%%%%%%%
\section{Logical Relations for Polymorphism}
We need to define denotation of universal type.
One possible definition would be:
\[
  \Denot{\forall\alpha\ldotp\tau} = \{ v \mid \forall \tau' \in\syncatset{Type}
    \ldotp v\;*\in \mathcal E \Denot{\Subst\tau{\tau'}x} \}
\]

Despite this definition looking promising, there is a severe problem,
because it is not structural -- type $\Subst\tau{\tau'}x$ could be structurally
bigger than $\tau$. We can try fixing this and add an environment $\Delta$ and
a function $\eta\colon\Delta\to\syncatset{Type}$ to remember type $\tau'$.
This would give us a definition
\begin{alignat*}{2}
  \Denot{\forall\alpha\ldotp\tau}_{\eta} & = \{ v \mid
    \forall \tau' \in \syncatset{Type}.
      v\;* \in \mathcal{E}\Denot{\tau}_{\eta[\alpha \mapsto \tau']} \} \\
  \Denot{\alpha}_{\eta} & = \Denot{\eta(\alpha)}
\end{alignat*}

This is close to proper definition, but it still isn't structural.
Trick that solves this problem is to instead of quantifying over
syntactic types, to do it over semantic types.
Therefore, we arrive at the correct definition, which is
\begin{alignat*}{2}
  \Denot{\forall\alpha\ldotp\tau}_{\eta} & = \{ v \mid
    \forall R \in \semcatset{Type}.
      v\;* \in \mathcal{E}\Denot{\tau}_{\eta[\alpha \mapsto R]} \} \\
  \Denot{\alpha}_{\eta} & = \eta(\alpha)
\end{alignat*}

For all the other denotations, we simply need to carry $\eta$.
For example the denotation of arrow would be defined as

\[
  \Denot{\tau_1 \to \tau_2}_\eta = \{ v \mid
    \forall v' \in \Denot{\tau_1}_\eta\ldotp
      v\;v' \in \mathcal{E}\Denot{\tau_2}_\eta \} \\\\
\]
Since we added a set $\Delta$ which is an environment for type variables,
we need to redefine logical relation from \autoref{sec:log-rel}.
The semantic representation for this type of environment is a function
$\eta\colon\Delta\to\semcatset{Type}$, that for each type variable returns
its meaning in terms of semantic types. Therefore, we have
\[
  \Delta; \Gamma\models e \;\colon\; \tau \iff
    \forall \eta\colon\Delta\to\semcatset{Type}\ldotp
      \forall \gamma\in\Denot\Gamma_\eta\ldotp
        \gamma^*\;e \in\Denot\tau_\eta
\]

%%%%%%%%%%%%%%%%%%%%%%%%%%%%%%%%%%%%%%%%%%%%%%%%%%%%%%%%%%%%%%%%%%%%%%%%%%%%%%%
\section{Further Reading}
