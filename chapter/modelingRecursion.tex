\chapter{Modeling Recursion}

%%%%%%%%%%%%%%%%%%%%%%%%%%%%%%%%%%%%%%%%%%%%%%%%%%%%%%%%%%%%%%%%%%%%%%%%%%%%%%%
\section{Kripke Semantics of Intuitionistic Logic}

In this section we will build a model in which
we can interpret intuitionistic logic.
Core idea standing behind this construction is that
truthfulness of formulas can only propagate to future worlds.
In other words things can become true as time moves on,
but not the other way around---if something is true it cannot become false.
This isn't the case in classical logic in which
we can assert that something is true and only later realize that we were wrong.

To capture this idea we can introduce a set $W$ of all worlds and
a relation $\sqsubseteq$ which will say
if a world is a future version of another one.
We need to ensure that $\sqsubseteq$ is a preorder meaning
its reflexive and transitive.

Similarly to classical logic we define valuation $\sigma$,
but in this case it also takes a world $w$ as it's argument.

\[
  \sigma \colon \Var \to W \to \{\mathsf{T}, \mathsf{F}\}\footnote{
    Instead of using $\{\mathsf{T}, \mathsf{F}\}$
    we could use any Heyting algebra.
  }
\]

In order to prevent true formulas from becoming false,
we need to ensure that for any $p\in\Var$
valuation $\sigma(p)$ is monotonic with respect to $\sqsubseteq$.

Now given valuation $\sigma$ and world $w$ we can define
when they \emph{satisfy} a formula $\varphi$
(denoted $\sigma, w \models \varphi$):

\begin{alignat*}{2}
  \sigma, w \models p & \iff \sigma(p)(w) = \mathsf{T} \\
  \sigma, w \models \varphi \wedge \psi & \iff
    \sigma, w \models \varphi \text{ and }\sigma, w \models \psi \\
  \sigma, w \models \top & \iff \text{always} \\
  \sigma, w \models \bot & \iff \text{never} \\
  \sigma, w \models \varphi \longrightarrow \psi & \iff
    \forall w'\sqsupseteq w \ldotp
      \sigma, w' \models \varphi \implies \sigma, w' \models \psi \\
\end{alignat*}

%%%%%%%%%%%%%%%%%%%%%%%%%%%%%%%%%%%%%%%%%%%%%%%%%%%%%%%%%%%%%%%%%%%%%%%%%%%%%%%
\section{Step-indexed Kripke Models}
\newcommand\Index[1]{|#1|}
\newcommand\Later{\triangleright}

\[
  (W, \sqsubseteq, \Index{\cdot})
\]

\[
  \Index{\cdot}\colon W \to \mathbb N
\]

$\Index{\cdot}$ should be monotonic, meaning
$\forall w \sqsupseteq w'\ldotp  \Index{w} \leq \Index{w'}$.

\[
  \varphi \Coloneqq \cdots \mid \Later \varphi
\]

\[
  \sigma, w \models \Later \varphi \iff
    \forall w' \sqsupseteq w\ldotp
      \Index{w'} + 1 \leq \Index{w} \implies \sigma, w' \models \varphi
\]

\begin{mathpar}
  \inferrule{\Sigma, \Gamma \vdash \varphi}
            {\triangleright \Sigma, \Gamma \vdash \Later \varphi}

  \inferrule{\Gamma, \Later \varphi \vdash \varphi}
            {\Gamma \vdash \varphi}
\end{mathpar}

%%%%%%%%%%%%%%%%%%%%%%%%%%%%%%%%%%%%%%%%%%%%%%%%%%%%%%%%%%%%%%%%%%%%%%%%%%%%%%%
\section{Recursive Types and Recursive Predicates}

%%%%%%%%%%%%%%%%%%%%%%%%%%%%%%%%%%%%%%%%%%%%%%%%%%%%%%%%%%%%%%%%%%%%%%%%%%%%%%%
\section{Banach Fixed-Point Theorem and Guarded Recursion}

%%%%%%%%%%%%%%%%%%%%%%%%%%%%%%%%%%%%%%%%%%%%%%%%%%%%%%%%%%%%%%%%%%%%%%%%%%%%%%%
\section{A Relational Model of Recursive Types}

Denotation of recursive types:
\[
  \Denot{\mu\alpha.\tau} = \{\texttt{fold}\;v \mid v \in\triangleright\Denot{\tau\{\mu\alpha.\tau/\alpha\}} \}
\]

or, alternatively:
\[
  \Denot{\mu\alpha.\tau}_{\eta} = \{\texttt{fold}\;v \mid v \in\triangleright\Denot{\tau}_{\eta[\alpha\mapsto \mu\alpha.\tau]} \}
\]

\[
  \Obs = \{ e \mid (\exists v.e=v) \lor \exists e' \in\triangleright\Obs \ldotp e \longrightarrow e' \}
\]

While those definitions use the thing being defined on right-hand side, all those recursive occurences are under $\triangleright$.
Therefore we may treat those as recursive predicates defined using fixed-point combinator:

\[
  \Obs = \mu(\lambda \Obs. \{ e \mid (\exists v.e=v) \lor \exists e' \in\triangleright\Obs \ldotp e \longrightarrow e' \})
\]
%%%%%%%%%%%%%%%%%%%%%%%%%%%%%%%%%%%%%%%%%%%%%%%%%%%%%%%%%%%%%%%%%%%%%%%%%%%%%%%
\section{Further Reading}
